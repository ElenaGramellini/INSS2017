\documentclass[a4paper]{article}

\usepackage[english]{babel}
\usepackage[utf8]{inputenc}
\usepackage{graphicx}
\usepackage{epsfig}
\usepackage{amsmath}
\usepackage{graphicx}
\usepackage{cancel}

%\usepackage{natbib} \setlength{\bibsep}{4.0pt}
\usepackage[colorinlistoftodos]{todonotes}
\usepackage{a4}
%\usepackage{cite}
%\usepackage{amssymb}
\usepackage{color}
\usepackage{lineno}
\usepackage{ulem}
\usepackage{enumerate}
\usepackage{comment}

\usepackage[left=2.5cm,right=2cm,top=2.5cm,bottom=2.cm]{geometry} 

%% for long url reference
\usepackage{hyperref}
\usepackage{url}
\makeatletter
\def\url@mystyle{%
  \@ifundefined{selectfont}{\def\UrlFont{\sf}}{\def\UrlFont{\small\ttfamily}}}
\makeatother
\urlstyle{my}



\renewcommand{\thefootnote}{\alph{footnote}}
\renewcommand{\topfraction}{.99}
\renewcommand{\bottomfraction}{.99}

\title{INSS 2017 Exercises}

%%%%%%%%%%%%%%%%%%%%%%%%%%%%%%%%%
\begin{document}
%%%%%%%%%%%%%%%%%%%%%%%%%%%%%%%%%
\def\Journal#1#2#3#4{{#1} {\bf #2}, #3 (#4)}
\def\etal{{\it et\ al.}}
\def\numunue{\nu_\mu\rightarrow\nu_e}
\def\numunutau{\nu_\mu\rightarrow\nu_\tau}
\def\nuebar{\bar\nu_e}
\def\nue{\nu_e}
\def\nutau{\nu_\tau}
\def\numubar{\bar\nu_\mu}
\def\numu{\nu_\mu}
\def\ra{\rightarrow}
\def\numubarnuebar{\bar\nu_\mu\rightarrow\bar\nu_e}
\def\nuebarnumubar{\bar\nu_e\rightarrow\bar\nu_\mu}
\def\osc{\rightsquigarrow}
\def\inteni{{\cal I}_{pot}}
\def\fmerit{{\cal F}}
%%%%%%%%%%%%%%%%%%%%%%%%%%%%%%%%%
\begin{flushright}
{\tt version 1.0}\\ 
\today
\end{flushright}
\vspace*{0.6cm}
%%%%%%%%%%%%%%%%%%%%%%%%%%%%%%%%%
\linenumbers
%%%%%%%%%%%%%%%%%%%%%%%%%%%%%%%%%
\begin{center}
{\Large \bf INSS 2017 Exercises} 
\vspace*{1.6cm}
\setcounter{footnote}{0}  
\def\A{\kern+.6ex\lower.42ex\hbox{$\scriptstyle \iota$}\kern-1.20ex a}
\def\E{\kern+.5ex\lower.42ex\hbox{$\scriptstyle \iota$}\kern-1.10ex e}
\small
\newcommand{\Aname}[2]{#1}
\def\titlefoot#1{\vspace{-0.3cm}\begin{center}{\bf #1}\end{center}}

Authors: Elena Gramellini\\

\end{center}
\vspace*{1cm}

\noindent Please send comments to: elena.gramellini@yale.edu


%%%%%%%%%%%%%%%%%%%%%%%%%%%%%%%%%
%% Table of content
%%%%%%%%%%%%%%%%%%%%%%%%%%%%%%%%%
\tableofcontents

\newpage
\section{Ex 1.1}
Charge pions decay almost 100\% of the times in $\pi\rightarrow\mu\nu_{\mu}$. \\
(1) In the rest frame of the pion, compute the muon energy as a function of the muon mass $m_{\mu}$, the pion mass $m_{\pi}$ and the neutrino mass $m_{\nu}$.\\
(2) What is the absolute value of the muon (tri)momentum?\\
(3) What is the relative change of muon momentum between $m_{\nu} = 0$ MeV and $m_{\nu} = 0.1$ MeV?\\

\subsection{Ex 1.1: answer}
For every massive particle :
\begin{equation} E^2 =  p^2 + m^2 \end{equation} 
cause if nature had to choose an invariant, she probably chose $c=1$.\\

Total energy and momentum in the pion rest frame before decay:
$$ (E_{\pi},  \vec{p}_{\pi}) = (m_{\pi}, 0)  $$

Total energy and momentum in the pion rest frame after decay:
$$ (E_{\nu}+E_{\mu}, \vec{p}_{\nu}+\vec{p}_{\mu}) = (m_{\pi}, 0)  $$
So in order to conserve energy we have:
\begin{equation} E_{\nu}+E_{\mu} = m_{\pi} \end{equation} 
and in order to conserve momentum we have:
$$ \vec{p}_{\nu}+\vec{p}_{\mu} = 0 .$$
The neutrino and the muon will decay back to back to conserve momentum. We'll just simply choose the axis of decay for this particle to reduce the problem to a one dimension.
So this last equation becomes 
$$ p_{\nu}+p_{\mu} = 0 $$ so
\begin{equation} p_{\nu} = -p_{\mu}  .\end{equation} 

Let's play a bit with equation (2)
$$ E_{\nu}+E_{\mu} = m_{\pi} \Rightarrow E_{\nu} = m_{\pi} - E_{\mu}$$ 
$$\Rightarrow$$ 
$$E^2_{\nu} = (m_{\pi} - E_{\mu})^2 $$ 
so using equation (1) and (3)
\begin{equation} E^2_{\nu} = p^2_{\nu} + m^2_{\nu} = p^2_{\mu} + m^2_{\nu}  = (m_{\pi} - E_{\mu})^2 = m^2_{\pi} - 2m_{\pi}E_{\mu} + E^2_{\mu}  \end{equation} 
$$E^2_{\mu} - m^2_{\mu} + m^2_{\nu}  = m^2_{\pi} - 2m_{\pi}E_{\mu} + E^2_{\mu}  $$
$$E^2_{\mu} - m^2_{\mu} + m^2_{\nu}  = m^2_{\pi} - 2m_{\pi}E_{\mu} + E^2_{\mu}  $$
$$ - m^2_{\mu} + m^2_{\nu}  = m^2_{\pi} - 2m_{\pi}E_{\mu}   $$
$$ 2m_{\pi}E_{\mu}  = m^2_{\pi} + m^2_{\mu} - m^2_{\nu} $$
So, as an answer to the first question, we have:
\begin{equation} E_{\mu}  = \frac{1}{2m_{\pi}}(m^2_{\pi} + m^2_{\mu} - m^2_{\nu}).  \end{equation} 
A couple of additional manipulation lead to an absolute value for the (tri)momentum of :
$$ E_{\mu} = \sqrt{p^2_{\mu}+m^2_{\mu}} = \frac{1}{2m_{\pi}}(m^2_{\pi} + m^2_{\mu} - m^2_{\nu}) $$
$$ p^2_{\mu}+m^2_{\mu} = \frac{1}{4m^2_{\pi}}(m^2_{\pi} + m^2_{\mu} - m^2_{\nu})^2 $$
$$ p^2_{\mu} = \frac{1}{4m^2_{\pi}}(m^2_{\pi} + m^2_{\mu} - m^2_{\nu})^2 - m^2_{\mu}$$
$$p_{\mu} = \sqrt{\frac{(m^2_{\pi} + m^2_{\mu} - m^2_{\nu})^2}{4m^2_{\pi}} - m^2_{\mu}}$$
\begin{equation} p_{\mu} = m_{\mu}\sqrt{\frac{(m^2_{\pi} + m^2_{\mu} - m^2_{\nu})^2}{4m^2_{\pi}m^2_{\mu}} - 1}\end{equation} 

So, the relative change in momentum from $m^2_{\nu} = 0$ and $m^2_{\nu} \neq 0$ is:
\begin{equation} \Delta p_{\mu}/p_{\mu} = \frac{p_{\mu,0} - p_{\mu,m_{\nu}}}{p_{\mu,m_{\nu} }} =  \frac{p_{\mu,0}}{p_{\mu,m_{\nu}}} -1 = 
\frac{m_{\mu}\sqrt{\frac{(m^2_{\pi} + m^2_{\mu})^2}{4m^2_{\pi}m^2_{\mu}} - 1} }{m_{\mu}\sqrt{\frac{(m^2_{\pi} + m^2_{\mu} - m^2_{\nu})^2}{4m^2_{\pi}m^2_{\mu}} - 1}} - 1\end{equation} 

\begin{equation} \Delta p_{\mu}/p_{\mu} 
\frac{\sqrt{\frac{(m^2_{\pi} + m^2_{\mu})^2}{4m^2_{\pi}m^2_{\mu}} - 1} }{\sqrt{\frac{(m^2_{\pi} + m^2_{\mu} - m^2_{\nu})^2}{4m^2_{\pi}m^2_{\mu}} - 1}} - 1\end{equation} 

\begin{equation} \Delta p_{\mu}/p_{\mu} =
\sqrt{\frac{(m^2_{\pi} + m^2_{\mu})^2 - 4m^2_{\pi}m^2_{\mu}  }{(m^2_{\pi} + m^2_{\mu} - m^2_{\nu})^2 - 4m^2_{\pi}m^2_{\mu}}} - 1\end{equation} 

\begin{equation} \Delta p_{\mu}/p_{\mu} =
\sqrt{\frac{(m^2_{\pi} - m^2_{\mu})^2 }{(m^2_{\pi} + m^2_{\mu} - m^2_{\nu})^2 - 4m^2_{\pi}m^2_{\mu}}} - 1\end{equation} 

\begin{equation} \Delta p_{\mu}/p_{\mu} =
\sqrt{\frac{(m^2_{\pi} - m^2_{\mu})^2 }{(m^2_{\pi} + m^2_{\mu})^2 -2(m^2_{\pi} + m^2_{\mu})m^2_{\nu}+ m^4_{\nu} - 4m^2_{\pi}m^2_{\mu}}} - 1\end{equation} 

\begin{equation} \Delta p_{\mu}/p_{\mu} =
\sqrt{\frac{(m^2_{\pi} - m^2_{\mu})^2 }{(m^2_{\pi} - m^2_{\mu})^2 -2(m^2_{\pi} + m^2_{\mu})m^2_{\nu}+ m^4_{\nu} }} - 1\end{equation} 



\begin{equation} \Delta p_{\mu}/p_{\mu} =
\sqrt{\frac{1}{1 -\frac{2(m^2_{\pi} + m^2_{\mu})m^2_{\nu}+ m^4_{\nu} } {(m^2_{\pi} - m^2_{\mu})^2}} } - 1\end{equation} 
We can taylor expand to
\begin{equation} \Delta p_{\mu}/p_{\mu} =
1 + \frac{2(m^2_{\pi} + m^2_{\mu})m^2_{\nu}+ m^4_{\nu} }{2(m^2_{\pi} - m^2_{\mu})^2} - 1\end{equation} 

\begin{equation} \Delta p_{\mu}/p_{\mu} =
\frac{2(m^2_{\pi} + m^2_{\mu})m^2_{\nu}+ m^4_{\nu} }{2(m^2_{\pi} - m^2_{\mu})^2} \end{equation} 

\begin{equation} \Delta p_{\mu}/p_{\mu} =
m^2_{\nu}\frac{(m^2_{\pi} + m^2_{\mu})}{{(m^2_{\pi} - m^2_{\mu})^2}} + \frac{m^4_{\nu} }{2(m^2_{\pi} - m^2_{\mu})^2} \end{equation} 


 
\end{document}